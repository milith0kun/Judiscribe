\documentclass[12pt,a4paper]{report}

% ============================================================================
% PAQUETES NECESARIOS
% ============================================================================
\usepackage[utf8]{inputenc}
\usepackage[spanish]{babel}
\usepackage{geometry}
\usepackage{graphicx}
\usepackage{hyperref}
\usepackage{fancyhdr}
\usepackage{tabularx}
\usepackage{longtable}
\usepackage{xcolor}
\usepackage{enumitem}
\usepackage{float}
\usepackage{booktabs}
\usepackage{multirow}
\usepackage{array}
\usepackage{colortbl}
\usepackage{caption}
\usepackage{titlesec}
\usepackage{tocloft}
\usepackage{amssymb}

% ============================================================================
% CONFIGURACIÓN DE MÁRGENES
% ============================================================================
\geometry{left=3cm, right=2.5cm, top=2.5cm, bottom=2.5cm}

% ============================================================================
% CONFIGURACIÓN DE ENCABEZADOS Y PIES DE PÁGINA
% ============================================================================
\setlength{\headheight}{14pt}
\pagestyle{fancy}
\fancyhf{}
\fancyhead[L]{\small JudiScribe -- Transcripción Judicial en Tiempo Real}
\fancyhead[R]{\small \thepage}
\renewcommand{\headrulewidth}{0.4pt}

% ============================================================================
% COLORES PERSONALIZADOS
% ============================================================================
\definecolor{primaryblue}{RGB}{0,102,204}
\definecolor{secondarygreen}{RGB}{0,153,76}
\definecolor{warningred}{RGB}{204,0,0}
\definecolor{darkgray}{RGB}{64,64,64}

% ============================================================================
% CONFIGURACIÓN DE HIPERVÍNCULOS
% ============================================================================
\hypersetup{
    colorlinks=true,
    linkcolor=primaryblue,
    filecolor=primaryblue,
    urlcolor=primaryblue,
    citecolor=primaryblue,
    pdftitle={JudiScribe - Documento de Captura de Requerimientos},
    pdfauthor={Proyecto Tesis},
    pdfsubject={Transcripción judicial, Distrito Judicial de Cusco},
    pdfkeywords={transcripción, judicial, Deepgram, TipTap, FastAPI, Next.js}
}

% ============================================================================
% CONFIGURACIÓN DE TÍTULOS DE SECCIONES
% ============================================================================
\titleformat{\chapter}[display]
  {\normalfont\huge\bfseries\color{primaryblue}}
  {\chaptertitlename\ \thechapter}{20pt}{\Huge}
\titleformat{\section}
  {\normalfont\Large\bfseries\color{primaryblue}}
  {\thesection}{1em}{}
\titleformat{\subsection}
  {\normalfont\large\bfseries\color{darkgray}}
  {\thesubsection}{1em}{}

% ============================================================================
% CONFIGURACIÓN DEL ÍNDICE
% ============================================================================
\renewcommand{\contentsname}{Índice General}
\setcounter{tocdepth}{3}
\setcounter{secnumdepth}{3}

% ============================================================================
% CONFIGURACIÓN DE TABLAS
% ============================================================================
\renewcommand{\arraystretch}{1.3}

% ============================================================================
% INICIO DEL DOCUMENTO
% ============================================================================
\begin{document}

% ============================================================================
% PORTADA
% ============================================================================
\begin{titlepage}
    \centering
    \vspace*{1.5cm}

    {\Huge\bfseries\color{primaryblue} DOCUMENTO DE CAPTURA\\[0.3cm] DE REQUERIMIENTOS\par}
    \vspace{1.5cm}

    {\LARGE\bfseries JudiScribe: Sistema de Transcripción\\[0.2cm] Judicial en Tiempo Real\par}
    \vspace{2.5cm}

    \begin{tabular}{rl}
        \textbf{PROYECTO:} & JudiScribe \\[0.3cm]
        \textbf{UBICACIÓN:} & Distrito Judicial de Cusco, Perú \\[0.3cm]
        \textbf{FECHA:} & Febrero de 2026 \\[0.3cm]
        \textbf{VERSIÓN:} & 1.0 \\[0.3cm]
    \end{tabular}

    \vspace{2.5cm}

    {\large\textbf{ELABORADO POR:}\par}
    \vspace{0.5cm}
    {\large Proyecto Tesis -- Ingeniería de Software\par}

    \vspace{2cm}

    {\large\textbf{MODELO DE IMPLEMENTACIÓN:}\par}
    \vspace{0.5cm}

    \begin{tabular}{ll}
        1 Nivel = 1 Mes = 5 Sprints & Producto mínimo viable \\
        3 Niveles = 3 Meses = 15 Sprints & Sistema completo operativo \\
    \end{tabular}

    \vfill

    {\large\textbf{\color{warningred} CLASIFICACIÓN: USO INTERNO}\par}

\end{titlepage}

% ============================================================================
% PÁGINA EN BLANCO DESPUÉS DE PORTADA
% ============================================================================
\newpage
\thispagestyle{empty}
\mbox{}

% ============================================================================
% ÍNDICE GENERAL
% ============================================================================
\newpage
\tableofcontents

% ============================================================================
% LISTA DE TABLAS
% ============================================================================
\newpage
\listoftables
\addcontentsline{toc}{chapter}{Lista de Tablas}

% ============================================================================
% INICIO DEL CONTENIDO PRINCIPAL
% ============================================================================
\newpage
\setcounter{page}{1}

% ============================================================================
% CAPÍTULO 1: RESUMEN EJECUTIVO
% ============================================================================
\chapter{Resumen Ejecutivo}

\section{Visión General del Proyecto}

JudiScribe es una aplicación web para transcriptores judiciales (digitadores) del Distrito Judicial de Cusco, Perú. El sistema permite transcribir audiencias judiciales en tiempo real usando inteligencia artificial y producir el acta oficial al finalizar.

\subsection{Descripción del Problema}

El habla humana ocurre a 150--180 palabras por minuto; un mecanógrafo rápido escribe 60--80, y las audiencias duran entre 2 y 5 horas. El digitador pierde información, se fatiga y después invierte horas adicionales formateando el acta. Los términos jurídicos técnicos se transcriben mal bajo presión.

\subsection{Solución Propuesta}

Un editor (Canvas) donde el texto aparece automáticamente mientras los participantes hablan, el digitador corrige lo que necesita, y al finalizar genera el acta con formato oficial del Poder Judicial peruano con un solo clic. Principio absoluto: el digitador siempre tiene el control; la IA nunca sobrescribe lo que el digitador ya editó.

\subsection{Componentes del Sistema}

\begin{itemize}
    \item \textbf{Backend (FastAPI):} WebSocket con Deepgram Nova-3, consolidación de segmentos, mejoramiento con Claude, diccionario jurídico (fuzzy matching), persistencia en PostgreSQL.
    \item \textbf{Frontend (Next.js 14 + TipTap):} Canvas de transcripción, panel de hablantes, reproductor de audio (wavesurfer.js), marcadores, sugerencias de corrección, generación de acta.
    \item \textbf{Infraestructura:} Docker Compose (backend, frontend, celery-worker, redis, postgres, nginx). Procesamiento batch con faster-whisper y Pyannote (GPU opcional).
\end{itemize}

\section{Modelo de Implementación por Niveles}

\begin{table}[H]
\centering
\begin{tabular}{|l|c|c|c|p{4.8cm}|p{2.8cm}|}
\hline
\rowcolor{primaryblue!20}
\textbf{Nivel} & \textbf{Mes} & \textbf{Sprints} & \textbf{Alcance} & \textbf{Entregables Clave} & \textbf{Inversión} \\
\hline
NIVEL 1 & Mes 1 & 1--5 & MVP & Detección multi-rol, estructura clara del documento, guardado de audios y transcripciones, Canvas, diccionario jurídico & S/. 1,500 \\
\hline
NIVEL 2 & Mes 2 & 6--10 & Extensión & Mejoras de precisión (no añadir palabras), batch, acta, exportación DOCX/PDF & S/. 1,500 -- 1,800 \\
\hline
NIVEL 3 & Mes 3 & 11--15 & Cierre & Usabilidad, capacitación, soporte post-lanzamiento & S/. 1,500 -- 1,800 \\
\hline
\rowcolor{secondarygreen!20}
\multicolumn{4}{|r|}{\textbf{Sistema completo:}} & \textbf{15 Sprints} & \textbf{S/. 4,500 -- 5,100} \\
\hline
\end{tabular}
\caption{Estructura de Niveles e Inversión del Proyecto JudiScribe}
\end{table}

El \textbf{Nivel 1} entrega el producto mínimo viable (MVP) con inversión de \textbf{S/. 1,500}: detección de múltiples roles (diarización), estructura clara del documento de audiencia, guardado correcto de audios (WAV) y transcripciones (segmentos en base de datos), Canvas operativo y sugerencias del diccionario jurídico. Todo el desarrollo es progresivo: cada nivel construye sobre el anterior.

\section{Beneficios Cuantificables}

\begin{table}[H]
\centering
\begin{tabular}{|p{4.2cm}|p{2.8cm}|p{2.8cm}|p{2.5cm}|}
\hline
\rowcolor{primaryblue!20}
\textbf{Beneficio} & \textbf{Situación Actual} & \textbf{Con JudiScribe} & \textbf{Mejora} \\
\hline
Velocidad de captura & 60--80 ppm (digitador) & 150--180 ppm (audio) & +100\% \\
\hline
Términos jurídicos mal transcritos & Frecuente & Sugerencias automáticas & Reducción de errores \\
\hline
Tiempo de formateo del acta & Horas manuales & Un clic (Claude) & -90\% \\
\hline
Trazabilidad audio--texto & Ninguna & Timestamps por segmento & 100\% \\
\hline
\end{tabular}
\caption{Impacto del Sistema en el Flujo del Digitador}
\end{table}

% ============================================================================
% CAPÍTULO 2: ANÁLISIS DE CONTEXTO Y STAKEHOLDERS
% ============================================================================
\chapter{Análisis de Contexto y Stakeholders}

\section{Contexto de Negocio}

\subsection{Organización}

El sistema está dirigido al Poder Judicial, Distrito Judicial de Cusco, Perú. Las audiencias pueden ser presenciales, virtuales (Google Meet) o mixtas. El audio se captura desde consola de audio, salida del sistema (getDisplayMedia) o archivo pregrabado.

\section{Análisis de Stakeholders}

\subsection{Digitador (Transcriptor)}

\textbf{Perfil:} Usuario principal. Asiste a la audiencia, supervisa la transcripción en tiempo real, corrige errores, inserta marcadores, asigna roles a los hablantes y genera el acta final. Nivel tecnológico medio: uso fluido de teclado y navegador.

\textbf{Necesidades:} Transcripción en tiempo real sin pérdida de contenido; detección de múltiples hablantes (multi-rol); sugerencias de corrección para términos jurídicos (solo reemplazo, sin añadir palabras); guardado fiable de audio y transcripción; estructura clara del documento de audiencia (encabezado, tipo, instancia); generación de acta con un clic y exportación a Word/PDF.

\textbf{Acceso:} Dashboard web (Canvas de transcripción, panel de hablantes, reproductor, marcadores, diccionario, generación de acta).

\subsection{Supervisor}

\textbf{Perfil:} Revisa actas generadas por los digitadores, hace correcciones menores y aprueba formalmente para que puedan exportarse como documento oficial.

\textbf{Necesidades:} Revisión del acta en el editor, aprobación con trazabilidad (quién y cuándo) y exportación DOCX/PDF.

\textbf{Acceso:} Mismo dashboard web con permisos de aprobación de acta.

\subsection{Administrador}

\textbf{Perfil:} Gestiona usuarios (solo el admin puede crear cuentas), configura el diccionario jurídico, keyterms para Deepgram y templates de actas.

\textbf{Necesidades:} CRUD de usuarios y asignación de roles; gestión del diccionario y keyterms; despliegue y variables de entorno.

\textbf{Acceso:} Dashboard web con sección de administración.

\section{Matriz RACI}

\begin{table}[H]
\centering
\small
\begin{tabular}{|p{4.5cm}|c|c|c|}
\hline
\rowcolor{primaryblue!20}
\textbf{Proceso} & \textbf{Digitador} & \textbf{Supervisor} & \textbf{Admin} \\
\hline
Iniciar/detener transcripción & R/A & I & -- \\
\hline
Corregir segmentos y aceptar sugerencias & R/A & I & -- \\
\hline
Asignar roles a hablantes & R/A & I & -- \\
\hline
Generar y editar acta & R/A & C & -- \\
\hline
Aprobar acta para exportación & I & R/A & -- \\
\hline
Exportar DOCX/PDF & R/A & I & -- \\
\hline
Gestionar diccionario y usuarios & -- & C & R/A \\
\hline
\end{tabular}
\caption{Matriz RACI -- JudiScribe}
\end{table}

% ============================================================================
% CAPÍTULO 3: ARQUITECTURA DEL SISTEMA
% ============================================================================
\chapter{Arquitectura del Sistema}

\section{Visión General}

El sistema se compone de: (1) Frontend Next.js 14 con TipTap y wavesurfer.js; (2) Backend FastAPI con WebSocket a Deepgram y servicios de mejoramiento; (3) PostgreSQL para audiencias, segmentos, hablantes, marcadores, actas y usuarios; (4) Redis y Celery para tareas batch; (5) Docker Compose y nginx como reverse proxy.

\section{Stack Tecnológico}

\begin{itemize}
    \item \textbf{Backend:} FastAPI, SQLAlchemy async, asyncpg, Deepgram Nova-3 (streaming), Claude Sonnet 4 (Anthropic), faster-whisper, Pyannote Audio (batch), Alembic.
    \item \textbf{Frontend:} Next.js 14, React 18, TypeScript, TipTap (ProseMirror), Zustand, Tailwind CSS, axios, wavesurfer.js.
    \item \textbf{Base de datos:} PostgreSQL 16 con JSONB.
    \item \textbf{Infraestructura:} Docker Compose (6 servicios), nginx, TLS.
\end{itemize}

\section{Modelo de Datos Principal}

Entidades principales: \textbf{audiencias} (expediente, juzgado, tipo, fecha, estado, audio\_path), \textbf{segmentos} (texto\_ia, texto\_mejorado, texto\_editado, editado\_por\_usuario, speaker\_id, timestamps, palabras\_json), \textbf{hablantes} (rol, etiqueta\_canvas, color), \textbf{marcadores}, \textbf{actas}, \textbf{usuarios}.

\section{Comunicación en Tiempo Real}

El cliente envía chunks de audio (base64 PCM 16\,kHz) por WebSocket. El backend reenvía a Deepgram y recibe transcripciones con speaker, confianza y palabras. Tras consolidación y mejoramiento (Claude), envía segmentos finales y sugerencias del diccionario jurídico al Canvas. El audio se graba en WAV en paralelo.

\section{Seguridad del Sistema}

Autenticación JWT (access y refresh token); validación de permiso por audiencia en el WebSocket. Contraseñas con bcrypt. Solo texto se envía a Claude, nunca audio. HTTPS en producción; nginx con terminación TLS. Variables sensibles (Deepgram, Anthropic) en entorno, no en código.

% ============================================================================
% CAPÍTULO 4: ROADMAP DE IMPLEMENTACIÓN
% ============================================================================
\chapter{Roadmap de Implementación}

\section{Filosofía de Desarrollo}

Desarrollo por sprints con 5 funcionalidades concretas por sprint. Cada sprint entrega software verificable. Se priorizan la transcripción en tiempo real (Nivel 1) y la generación de acta (Nivel 2).

\section{Nivel 1 (Mes 1): Producto Mínimo Viable}

\textbf{Inversión:} S/. 1,500 \\
\textbf{Duración:} 4 semanas (5 sprints). \\
\textbf{Objetivo:} Entregar el MVP con (1) \textbf{detección multi-rol} (diarización por Deepgram + panel de hablantes con asignación de roles), (2) \textbf{estructura clara del documento} de audiencia (encabezado completo, tipo e instancia, plantilla según tipo), (3) \textbf{guardado correcto de audios y transcripciones} (archivo WAV por audiencia y todos los segmentos persistidos en base de datos con integridad), (4) Canvas operativo con sugerencias del diccionario jurídico. Todo progresivo para que la primera etapa sea un producto mínimo viable funcional.

\subsection{Sprint 1: Infraestructura, Base de Datos y Estructura del Documento}

\textbf{Semana 1}

\subsubsection{Funcionalidad 1: Configuración del Backend y Base de Datos}

\begin{itemize}
    \item Proyecto FastAPI con estructura modular (api, ws, services, models, schemas).
    \item PostgreSQL 16 con SQLAlchemy async y asyncpg.
    \item Modelos: Audiencia, Segmento, Usuario, Hablante; migraciones con Alembic.
    \item Variables de entorno con pydantic-settings (config.py).
\end{itemize}

\subsubsection{Funcionalidad 2: Autenticación y Roles}

\begin{itemize}
    \item Endpoints de login y refresh con JWT (access + refresh).
    \item Roles: admin, transcriptor, supervisor. Guards por rol.
    \item Bcrypt para contraseñas. Tabla usuarios con email, nombre, rol, activo.
\end{itemize}

\subsubsection{Funcionalidad 3: Frontend Next.js Base}

\begin{itemize}
    \item Next.js 14 con App Router, TypeScript, Tailwind CSS.
    \item Pantallas de Login y layout principal con rutas protegidas.
    \item AuthProvider y AuthGuard. Zustand para estado de autenticación.
\end{itemize}

\subsubsection{Funcionalidad 4: API REST de Audiencias y Estructura Clara del Documento}

\begin{itemize}
    \item CRUD de audiencias con \textbf{estructura completa del documento}: expediente, juzgado, tipo\_audiencia (juicio oral, prisión preventiva, apelación, etc.), instancia (juzgado\_unipersonal, sala\_apelaciones), fecha, hora\_inicio, sala, delito, imputado\_nombre, agraviado\_nombre, especialista\_causa, especialista\_audiencia, estado.
    \item Listar y filtrar por fecha, juzgado, estado. El documento de audiencia tiene una estructura clara y predecible para transcripción y acta.
    \item Opcional en Sprint 1: carga de plantilla o secciones por tipo de audiencia (templates\_audiencia) para preestructurar el Canvas según el tipo.
\end{itemize}

\subsubsection{Funcionalidad 5: Integración WebSocket con Deepgram y Detección Multi-Rol}

\begin{itemize}
    \item Servicio DeepgramStreamingService: conexión WSS a Deepgram Nova-3 (es-419, \textbf{diarize=true} para detección de múltiples hablantes, smart\_format, keyterms).
    \item Cada resultado incluye \texttt{speaker} (SPEAKER\_00, SPEAKER\_01, etc.): base para \textbf{detección multi-rol} en tiempo real.
    \item Endpoint WebSocket /ws/transcripcion/\{audiencia\_id\} con token JWT en query. Recepción de resultados (transcript, speaker, confidence, words) y callback al handler.
\end{itemize}

\subsection{Sprint 2: Guardado de Audios y Transcripciones}

\textbf{Semana 2}

Este sprint garantiza que \textbf{audios y transcripciones se guarden correctamente}: un archivo WAV por audiencia y todos los segmentos en base de datos con trazabilidad e integridad.

\subsubsection{Funcionalidad 1: Captura de Audio en el Navegador}

\begin{itemize}
    \item Hook useAudioCapture: MediaRecorder a 16\,kHz mono, chunks cada 250\,ms.
    \item Envío por WebSocket (type: audio\_chunk, data: base64). Selección de fuente (micrófono, audio del sistema).
\end{itemize}

\subsubsection{Funcionalidad 2: Grabación WAV en Servidor (Guardado Correcto de Audio)}

\begin{itemize}
    \item En transcription\_ws: apertura de archivo WAV (16\,bit, 16\,kHz, mono) por audiencia en ruta persistente (ej. \texttt{AUDIO\_STORAGE\_PATH}).
    \item Escritura de frames por cada chunk recibido. Cierre correcto del archivo y actualización en BD de \texttt{audio\_path} y \texttt{audio\_duration\_seconds} al finalizar la sesión.
    \item El audio queda guardado de forma confiable para consulta posterior y para procesamiento batch (Nivel 2).
\end{itemize}

\subsubsection{Funcionalidad 3: Modelo y Persistencia de Segmentos (Guardado Correcto de Transcripciones)}

\begin{itemize}
    \item Modelo Segmento: audiencia\_id, speaker\_id, texto\_ia, texto\_mejorado, timestamp\_inicio/fin, confianza, orden, palabras\_json, editado\_por\_usuario, fuente.
    \item \textbf{Guardar cada segmento final} en PostgreSQL desde el handler WebSocket con orden secuencial y timestamps; sin pérdida de segmentos. API de segmentos para recuperar la transcripción completa de la audiencia.
\end{itemize}

\subsubsection{Funcionalidad 4: Consolidación de Segmentos por Speaker}

\begin{itemize}
    \item Buffer de consolidación en transcription\_ws: acumular segmentos del mismo speaker hasta frase completa.
    \item Reglas de completitud: palabras conectoras al final, respuestas cortas válidas (sí, no, niego, etc.), opcional chequeo con Claude.
\end{itemize}

\subsubsection{Funcionalidad 5: Envío de Segmentos al Cliente}

\begin{itemize}
    \item Mensajes JSON al frontend: type transcript, is\_final, speaker, text, texto\_mejorado, confidence, start, end, words.
    \item Envío de provisionales (is\_final=false) sin guardar en BD.
\end{itemize}

\subsection{Sprint 3: Mejoramiento en Tiempo Real y Diccionario Jurídico}

\textbf{Semana 2--3}

Este sprint incorpora el mejoramiento del texto con Claude y las sugerencias del diccionario jurídico para reducir el margen de error en términos legales.

\subsubsection{Funcionalidad 1: Servicio de Mejoramiento en Tiempo Real (Claude)}

\begin{itemize}
    \item RealTimeEnhancementService: método enhance\_segment(texto, speaker\_id, segmentos\_previos).
    \item Prompt a Claude Sonnet 4: mejorar puntuación, mayúsculas (Juez, Fiscal, Señor), signos de interrogación, sin inventar contenido.
    \item Retorno: texto mejorado, is\_question, confidence. Contexto de últimos 25 segmentos.
\end{itemize}

\subsubsection{Funcionalidad 2: Detección de Completitud de Frase}

\begin{itemize}
    \item Método is\_sentence\_complete: consulta a Claude si el buffer actual es frase completa o debe esperar más texto.
    \item Uso cuando el buffer termina en conector y tiene más de 20 palabras; evita cortar oraciones a la mitad.
\end{itemize}

\subsubsection{Funcionalidad 3: Diccionario Jurídico con Fuzzy Matching}

\begin{itemize}
    \item LegalDictionary: carga de legal\_terms.json (correcto, variantes\_error, categoria, contexto).
    \item Soundex adaptado al español y Levenshtein para búsqueda fuzzy. check\_segment(texto) devuelve lista de Correction (original, suggested, confidence, position).
    \item Objetivo: menos de 50\,ms por segmento. Envío de sugerencias por WebSocket (type: suggestion) tras cada segmento mejorado.
\end{itemize}

\subsubsection{Funcionalidad 4: Keyterms para Deepgram}

\begin{itemize}
    \item legal\_keyterms.py: listas por categoría (procesal penal, delitos, partes procesales, normativa, regional Cusco).
    \item get\_keyterms(100): términos únicos enviados en la URL de Deepgram para mejorar reconocimiento de vocabulario jurídico.
\end{itemize}

\subsubsection{Funcionalidad 5: Integración Mejoramiento + Diccionario en el Flujo}

\begin{itemize}
    \item Tras consolidar segmento: llamar enhance\_segment; enviar texto\_mejorado al cliente; ejecutar dictionary.check\_segment(texto\_mejorado) y enviar sugerencias.
    \item Guardar en BD texto\_ia y texto\_mejorado. No modificar segmentos con editado\_por\_usuario=true.
\end{itemize}

\textbf{Criterios de Aceptación Sprint 3:}
\begin{itemize}
    \item Cada segmento final mostrado en el Canvas tiene puntuación y mayúsculas coherentes con el contexto judicial.
    \item Las sugerencias del diccionario (ej. ``sobresemiento'' $\to$ ``sobreseimiento'') se envían al cliente y se muestran en el Canvas.
    \item No se cortan frases a la mitad por timeout; la consolidación espera hasta frase completa cuando es posible.
\end{itemize}

\subsection{Sprint 4: Canvas TipTap y Panel de Hablantes}

\textbf{Semana 3}

\subsubsection{Funcionalidad 1: Editor TipTap con Extensiones}

\begin{itemize}
    \item TranscriptionCanvas: editor con SpeakerNode, SegmentMark, LowConfidenceMark, ProvisionalNode.
    \item Inserción automática de nodos de hablante al cambiar speaker. Estilos para texto provisional (gris) y baja confianza (amarillo).
\end{itemize}

\subsubsection{Funcionalidad 2: Panel de Hablantes}

\begin{itemize}
    \item PanelHablantes: lista de SPEAKER\_00, SPEAKER\_01, etc. con selector de rol (Juez, Fiscal, Defensa, Imputado, etc.) y etiqueta en Canvas.
    \item API hablantes: GET/PUT por audiencia y speaker\_id. Colores por hablante.
\end{itemize}

\subsubsection{Funcionalidad 3: Reproductor de Audio}

\begin{itemize}
    \item ReproductorAudio con wavesurfer.js: forma de onda, play/pause, velocidad, salto por timestamp.
    \item Sincronización: clic en segmento del Canvas mueve el reproductor al timestamp\_inicio del segmento.
\end{itemize}

\subsubsection{Funcionalidad 4: Marcadores y Panel de Marcadores}

\begin{itemize}
    \item BookmarkNode en TipTap; creación de marcador con nota y timestamp. API marcadores: POST, GET, DELETE.
    \item PanelMarcadores: lista cronológica; clic navega al punto en Canvas y en el audio.
\end{itemize}

\subsubsection{Funcionalidad 5: Sugerencias en el Canvas}

\begin{itemize}
    \item SuggestionPopover: muestra corrección del diccionario (original $\to$ sugerido); Tab aceptar, Esc rechazar.
    \item WordCorrectionPopover: análisis contextual con Claude (analyzeWordInContext) para palabras de baja confianza; sugerencias de reemplazo 1:1 sin añadir palabras.
\end{itemize}

\subsection{Sprint 5: Estabilización, Frases Estándar y Puesta en Producción}

\textbf{Semana 4}

Este sprint cierra el MVP con atajos de frases estándar, barra de estado, generación de acta básica y despliegue.

\subsubsection{Funcionalidad 1: Frases Estándar y Atajos}

\begin{itemize}
    \item Modelo frase\_estandar: usuario\_id (nullable para globales), texto, descripcion, atajo, orden.
    \item API frases: GET/POST/PUT/DELETE. Componente AtajosFrases: menú de frases con atajo de teclado; inserción en la posición del cursor en el Canvas.
    \item Ejemplos: ``Sí, Señoría.'', ``No, Señoría.'', ``Protesto, Señoría.'', frases de apertura/cierre de audiencia.
\end{itemize}

\subsubsection{Funcionalidad 2: Predicción y Análisis Contextual}

\begin{itemize}
    \item Servicio text\_prediction: predict\_completion (contexto con Claude), detect\_names, detect\_expediente, capitalize\_proper\_nouns.
    \item API /prediction/suggest y /analysis/word, /analysis/phrase para corrección contextual (solo reemplazo de palabras, sin añadir texto).
\end{itemize}

\subsubsection{Funcionalidad 3: Barra de Estado y Persistencia de Ediciones}

\begin{itemize}
    \item BarraEstado: conteo de palabras, tiempo transcurrido, estado de conexión Deepgram (Conectado/Reconectando/Desconectado), número de segmentos.
    \item PUT segmentos: guardar texto editado por el digitador; marcar editado\_por\_usuario=true para aplicar regla de no-sobreescritura.
\end{itemize}

\subsubsection{Funcionalidad 4: Generación de Acta (Borrador)}

\begin{itemize}
    \item Endpoint POST generar-acta: snapshot del contenido del Canvas + metadatos de audiencia y hablantes.
    \item Envío a Claude Sonnet 4 con prompt de formato oficial (Formato A unipersonal / Formato B apelaciones). Respuesta en acta.contenido\_llm; versión y estado borrador.
\end{itemize}

\subsubsection{Funcionalidad 5: Docker Compose y Despliegue}

\begin{itemize}
    \item Docker Compose: backend (FastAPI), frontend (Next.js), postgres, redis, nginx. Variables de entorno documentadas.
    \item nginx como reverse proxy con timeout largo para WebSocket de transcripción (ej. 3600\,s).
    \item Documentación de despliegue y credenciales (Deepgram, Anthropic) en .env.
\end{itemize}

\textbf{Criterios de Aceptación Sprint 5:}
\begin{itemize}
    \item El digitador puede usar atajos de frases estándar desde el teclado.
    \item Las ediciones del digitador se persisten y no son sobrescritas por la IA.
    \item Se puede generar un borrador de acta desde el Canvas y visualizarlo en el editor de acta.
    \item El sistema se despliega con Docker Compose y la transcripción en vivo funciona de extremo a extremo.
\end{itemize}

\textbf{Estado al Finalizar Nivel 1 (MVP):}
\begin{itemize}
    \item Transcripción en tiempo real con Deepgram, consolidación por speaker y mejoramiento con Claude.
    \item Canvas con hablantes, marcadores, sugerencias del diccionario y corrección contextual (reemplazo 1:1).
    \item Audio grabado en WAV; reproductor sincronizado con el Canvas.
    \item Generación de borrador de acta con Claude. Sistema desplegado y operativo.
\end{itemize}

\section{Nivel 2 (Mes 2): Mejoras de Precisión, Batch, Acta y Exportación}

\textbf{Inversión:} S/. 1,500 -- S/. 1,800 \\
\textbf{Duración:} 4 semanas. \textbf{Objetivo:} (1) \textbf{Revisión de precisión} según el plan de mejoras: reglas ``no añadir palabras'', autocompletado solo como corrección (reemplazo 1:1), umbrales de confianza; (2) Procesamiento batch opcional (faster-whisper + Pyannote) y revisión de propuestas; (3) Acta editable y aprobación; (4) Exportación DOCX y PDF; (5) Corpus e indexación para que el sistema aprenda (Sprint 12 en Nivel 3). La distribución es progresiva: primero se afianzan las reglas de precisión y luego se añade batch, acta y exportación.

\subsection{Sprint 6: Mejoras de Precisión y Reglas de No-Añadir Palabras}

\textbf{Semana 1}

Este sprint implementa las mejoras del plan de revisión para minimizar el margen de error y alinear el sistema a ``solo corrección, sin añadir palabras''.

\subsubsection{Funcionalidad 1: Regla Explícita en Claude (No Añadir Palabras)}

\begin{itemize}
    \item En el prompt de \texttt{enhance\_segment} (RealTimeEnhancementService): añadir y priorizar la regla \textbf{``NUNCA añadir palabras ni completar frases. Solo: puntuación, mayúsculas, y reemplazo de palabra mal transcrita por la forma correcta (1 palabra $\to$ 1 palabra).''}
    \item Revisar \texttt{context\_analysis.get\_phrase\_corrections}: quitar ``completar frases''; sustituir por ``solo corregir palabras existentes (ortografía, tildes, términos legales); no añadir ni completar texto''.
\end{itemize}

\subsubsection{Funcionalidad 2: Autocompletado Solo como Corrección (Reemplazo 1:1)}

\begin{itemize}
    \item Desactivar o reconvertir InlineSuggestion para que \textbf{no} inserte texto de continuación (ghost text); solo sugerir \textbf{reemplazo} de la palabra bajo el cursor cuando aplique.
    \item Mantener WordCorrectionPopover y SuggestionPopover como flujo principal: sugerencias de reemplazo (diccionario legal, alternativas Deepgram, IA contextual) con Tab/Esc y 1--5 para elegir. Ninguna sugerencia debe añadir palabras nuevas al texto.
\end{itemize}

\subsubsection{Funcionalidad 3: Umbrales de Confianza y Falsos Positivos}

\begin{itemize}
    \item Usar umbral de confianza (ej. 0.85) en la UI para mostrar ``reemplazo sugerido'' solo en palabras con confidence $<$ 0.85. Excluir de sugerencias las palabras que ya figuren como correctas en \texttt{legal\_terms}.
    \item Documentar y, si aplica, ajustar \texttt{utterance\_end\_ms} / \texttt{endpointing} en Deepgram para reducir cortes de frase.
\end{itemize}

\subsubsection{Funcionalidad 4: Consolidación y Completitud de Frase}

\begin{itemize}
    \item Reforzar reglas de consolidación: no procesar como segmento final bloques de menos de 5 palabras salvo respuestas cortas válidas (sí, no, niego, etc.). Mantener uso de Claude para \texttt{is\_sentence\_complete} cuando el buffer termina en conector y tiene $>$ 20 palabras.
\end{itemize}

\subsubsection{Funcionalidad 5: API de Sugerencia de Reemplazo (Opcional)}

\begin{itemize}
    \item Endpoint o modo en \texttt{/prediction/suggest} para ``sugerencia de reemplazo para la palabra actual'' (contexto = frase + palabra seleccionada), devolviendo una palabra o lista de candidatos, no continuación de frase.
\end{itemize}

\textbf{Criterios de Aceptación Sprint 6:}
\begin{itemize}
    \item Claude no añade palabras en el texto mejorado; solo puntuación, mayúsculas y sustituciones 1:1.
    \item Las sugerencias en el Canvas son solo de reemplazo; no hay inserción de texto nuevo al aceptar.
    \item Las sugerencias del diccionario se muestran con umbral de confianza coherente y sin falsos positivos evidentes.
\end{itemize}

\subsection{Sprint 7: Procesamiento Batch de Audio}

\textbf{Semana 1}

\subsubsection{Funcionalidad 1: Tarea Celery batch\_process\_audio}

\begin{itemize}
    \item Tarea Celery \texttt{batch\_process\_audio(audiencia\_id)} disparada al finalizar la transcripción en vivo o al subir audio pregrabado.
    \item Cola Redis; reintentos (max\_retries=3) en caso de fallo. Logs de progreso para monitoreo.
\end{itemize}

\subsubsection{Funcionalidad 2: Transcripción con faster-whisper large-v3}

\begin{itemize}
    \item Transcripción del archivo WAV completo con faster-whisper (modelo large-v3, beam\_size=5, word\_timestamps=True, language=es).
    \item Initial prompt con terminología judicial y keyterms de Cusco para mejorar precisión. Salida: segmentos con texto y timestamps por palabra.
\end{itemize}

\subsubsection{Funcionalidad 3: Diarización con Pyannote 3.1}

\begin{itemize}
    \item Pipeline Pyannote sobre el mismo audio: identificación de speakers (SPEAKER\_00, SPEAKER\_01, etc.) con timestamps de inicio y fin de cada turno.
    \item Requiere GPU (NVIDIA RTX 3060 12GB mínimo) o ejecución en servicio externo.
\end{itemize}

\subsubsection{Funcionalidad 4: Alineación temporal transcripción--diarización}

\begin{itemize}
    \item Servicio de alineación: cruce por superposición temporal entre palabras de faster-whisper y segmentos de Pyannote.
    \item Asignación de speaker\_id a cada palabra/segmento de la transcripción batch. Fusión en segmentos coherentes con texto y speaker.
\end{itemize}

\subsubsection{Funcionalidad 5: Propuestas de mejora sin tocar editado\_por\_usuario}

\begin{itemize}
    \item Comparación resultado batch vs segmentos ya guardados (streaming). Para cada segmento: si \texttt{editado\_por\_usuario=true}, no se genera propuesta.
    \item Si la diferencia (Levenshtein o similar) supera umbral configurable, se crea propuesta de mejora para que el digitador acepte o rechace en la vista de revisión.
\end{itemize}

\subsection{Sprint 8: Revisión de Propuestas y Merge}

\textbf{Semana 2}

\subsubsection{Funcionalidad 1: Vista de comparación batch vs streaming}

\begin{itemize}
    \item Pantalla de revisión post-batch: lista de segmentos con texto actual (Canvas) y texto propuesto (batch) lado a lado.
    \item Resaltado de diferencias por palabra o por bloque para facilitar la decisión.
\end{itemize}

\subsubsection{Funcionalidad 2: Aceptar o rechazar por segmento}

\begin{itemize}
    \item Botones o atajos por cada propuesta: Aceptar (reemplazar segmento en Canvas por versión batch) o Rechazar (mantener texto actual).
    \item Solo se actualizan segmentos no editados por usuario; los editados se muestran como ``Sin propuesta'' o bloqueados.
\end{itemize}

\subsubsection{Funcionalidad 3: API batch-update}

\begin{itemize}
    \item Endpoint POST \texttt{/api/audiencias/\{id\}/segmentos/batch-update}: recibe lista de decisiones (segment\_id, accion: aceptar|rechazar).
    \item Actualización en BD de texto\_ia/texto\_mejorado cuando se acepta; persistencia del rechazo para no volver a proponer.
\end{itemize}

\subsubsection{Funcionalidad 4: Actualización del Canvas}

\begin{itemize}
    \item Tras aceptar propuestas, el Canvas se actualiza con el nuevo contenido sin recargar la página. Sincronización con store (Zustand) y con el servidor.
\end{itemize}

\subsubsection{Funcionalidad 5: Historial de cambios batch}

\begin{itemize}
    \item Registro en audit\_log o tabla dedicada: qué segmentos fueron actualizados por batch, fecha y usuario que aprobó. Consulta opcional desde el panel de auditoría.
\end{itemize}

\subsection{Sprint 9: Editor de Acta y Aprobación}

\textbf{Semana 3}

\subsubsection{Funcionalidad 1: ActaEditor TipTap}

\begin{itemize}
    \item Editor TipTap dedicado para el acta: carga \texttt{contenido\_llm} o \texttt{contenido\_final} según estado. Edición libre por el digitador o supervisor.
    \item Estilos y estructura (títulos, párrafos, tablas) según formato oficial del Poder Judicial.
\end{itemize}

\subsubsection{Funcionalidad 2: Guardado de contenido\_final}

\begin{itemize}
    \item PUT \texttt{/api/audiencias/\{id\}/acta}: guarda el contenido editado en \texttt{acta.contenido\_final}. Versión se mantiene o se incrementa según política (ej. solo al regenerar con Claude).
\end{itemize}

\subsubsection{Funcionalidad 3: Versionado de actas}

\begin{itemize}
    \item Tabla actas con campo \texttt{version}; GET \texttt{/acta/versiones} devuelve historial de versiones con fecha y usuario.
    \item Posibilidad de ver versión anterior (solo lectura) para comparar.
\end{itemize}

\subsubsection{Funcionalidad 4: Flujo aprobar acta (supervisor)}

\begin{itemize}
    \item Endpoint POST \texttt{/api/audiencias/\{id\}/acta/aprobar}: solo rol supervisor (o admin). Cambia estado a ``aprobada'', registra \texttt{aprobada\_por} y \texttt{aprobada\_at}.
    \item Botón ``Aprobar acta'' visible solo para supervisores; el digitador puede editar y solicitar revisión.
\end{itemize}

\subsubsection{Funcionalidad 5: Estados borrador, en\_revision, aprobada, exportada}

\begin{itemize}
    \item Máquina de estados: borrador $\to$ en\_revision $\to$ aprobada $\to$ exportada. Transiciones validadas por rol. Exportación DOCX/PDF permitida cuando estado es aprobada (o según política: también desde en\_revision para borrador).
\end{itemize}

\subsection{Sprint 10: Exportación DOCX/PDF y Estabilización Nivel 2}

\textbf{Semana 4}

\subsubsection{Funcionalidad 1: Plantilla de acta (python-docx)}

\begin{itemize}
    \item Plantilla Word con estilos oficiales (encabezado, títulos, cuerpo, firmas). Datos de audiencia (expediente, juzgado, fecha, participantes) y contenido del acta insertados programáticamente.
\end{itemize}

\subsubsection{Funcionalidad 2: Exportar DOCX con formato oficial}

\begin{itemize}
    \item POST \texttt{/api/audiencias/\{id\}/exportar/docx}: genera archivo .docx con python-docx y devuelve enlace de descarga o stream. Nombre de archivo: expediente\_acta.docx.
\end{itemize}

\subsubsection{Funcionalidad 3: Exportar PDF}

\begin{itemize}
    \item POST \texttt{/api/audiencias/\{id\}/exportar/pdf}: generación vía weasyprint u otra librería (HTML a PDF) con la misma estructura que el acta. Descarga directa.
\end{itemize}

\subsubsection{Funcionalidad 4: Descarga por audiencia}

\begin{itemize}
    \item Solo audiencias con acta en estado aprobada (o permitir borrador según configuración). Control de permisos: solo quien tenga acceso a la audiencia puede exportar.
\end{itemize}

\subsubsection{Funcionalidad 5: Estabilización y cierre de Nivel 2}

\begin{itemize}
    \item Pruebas E2E del flujo completo: audiencia $\to$ transcripción $\to$ edición $\to$ acta $\to$ aprobar $\to$ exportar DOCX/PDF.
    \item Corrección de bugs reportados; documentación de usuario para digitadores y supervisores; capacitación (2--3 h); checklist de aceptación de Nivel 2 (batch, acta, aprobación, exportación).
    \item Cada exportación se registra en audit\_log (usuario, audiencia\_id, formato, timestamp).
\end{itemize}

\section{Nivel 3 (Mes 3): Cierre y Soporte}

\textbf{Duración:} 4 semanas. \textbf{Objetivo:} Mejoras de usabilidad, corpus y sugerencias (opcional), capacitación final y soporte post-lanzamiento.

\subsection{Sprint 11: Mejoras de UI/UX y Rendimiento}

\textbf{Semana 1}

\begin{itemize}
    \item \textbf{Funcionalidad 1:} Mejoras de interfaz solicitadas tras uso real (feedback de digitadores): disposición del panel, tamaños de fuente, contraste.
    \item \textbf{Funcionalidad 2:} Optimización del rendimiento del Canvas (documentos largos): virtualización o paginación si aplica; reducción de re-renders.
    \item \textbf{Funcionalidad 3:} Accesibilidad: contraste WCAG, navegación por teclado, etiquetas ARIA en componentes críticos.
    \item \textbf{Funcionalidad 4:} Atajos de teclado documentados y consistentes (Tab, Esc, Ctrl+J para auto-scroll, etc.).
    \item \textbf{Funcionalidad 5:} Ajustes en el flujo de generación de acta (mensajes de progreso, manejo de errores de Claude).
\end{itemize}

\subsection{Sprint 12: Corpus y Aprendizaje del Sistema}

\textbf{Semana 2}

\begin{itemize}
    \item \textbf{Funcionalidad 1:} Extracción de corpus: lectura de audiencias y segmentos (texto\_ia, texto\_mejorado, texto\_editado) con criterio de verdad (editado $>$ mejorado $>$ ia).
    \item \textbf{Funcionalidad 2:} Tokenización y normalización del texto; construcción de índice de frecuencias de términos.
    \item \textbf{Funcionalidad 3:} Extracción de pares error--correcto (diferencias entre texto\_ia y texto editado/mejorado) para alimentar variantes del diccionario.
    \item \textbf{Funcionalidad 4:} Actualización de keyterms para Deepgram desde los términos más frecuentes del corpus (respetando límite de 100).
    \item \textbf{Funcionalidad 5:} Tarea Celery o script periódico para reconstruir corpus e índices; uso solo con consentimiento o política de datos definida.
\end{itemize}

\subsection{Sprints 13--15: Cierre del Proyecto}

\subsection*{Sprint 13: Ajustes finales y documentación técnica}

\begin{itemize}
    \item Correcciones menores post-capacitación. Documentación técnica: arquitectura, API, despliegue, variables de entorno. Entrega de código fuente y repositorio.
\end{itemize}

\subsection*{Sprint 14: Capacitación completa y entrega}

\begin{itemize}
    \item Capacitación final a todos los roles (digitadores, supervisores, administrador). Entrega de credenciales, manuales y acceso a documentación. Firma de acta de entrega si aplica.
\end{itemize}

\subsection*{Sprint 15: Soporte post-lanzamiento}

\begin{itemize}
    \item Soporte intensivo durante la primera semana de uso en producción: resolución de incidencias, ajustes de configuración, monitoreo de estabilidad.
\end{itemize}

% ============================================================================
% CAPÍTULO 5: ESPECIFICACIÓN FUNCIONAL DETALLADA
% ============================================================================
\chapter{Especificación Funcional Detallada}

\section{Módulo: Transcripción en Tiempo Real}

\subsection{Flujo General}

Crear audiencia $\to$ iniciar transcripción $\to$ captura de audio (16\,kHz mono) $\to$ envío por WebSocket $\to$ Deepgram Nova-3 $\to$ consolidación por speaker $\to$ Claude (mejora) $\to$ diccionario (sugerencias) $\to$ envío al Canvas $\to$ persistencia de segmentos en PostgreSQL.

\subsection{Regla de No-Sobreescritura}

Si \texttt{editado\_por\_usuario=true}, ningún proceso automático (streaming, batch, LLM) puede modificar ese segmento. El texto nuevo de la IA siempre se agrega al final del Canvas; nunca reemplaza contenido ya editado por el digitador.

\subsection{Datos del Segmento}

\begin{table}[H]
\centering
\small
\begin{tabular}{|p{3.2cm}|p{2.2cm}|p{1.8cm}|p{5cm}|}
\hline
\rowcolor{primaryblue!20}
\textbf{Campo} & \textbf{Tipo} & \textbf{Oblig.} & \textbf{Descripción} \\
\hline
audiencia\_id & UUID & Sí & Audiencia a la que pertenece \\
\hline
speaker\_id & VARCHAR(20) & Sí & SPEAKER\_00, SPEAKER\_01, etc. \\
\hline
texto\_ia & TEXT & Sí & Texto original de Deepgram/Whisper \\
\hline
texto\_mejorado & TEXT & No & Texto tras Claude (puntuación, mayúsculas) \\
\hline
texto\_editado & TEXT & No & Texto corregido por el digitador \\
\hline
timestamp\_inicio/fin & FLOAT & Sí & Segundos en el audio \\
\hline
confianza & FLOAT & Sí & Score ASR (0--1) \\
\hline
editado\_por\_usuario & BOOLEAN & Sí & Si el digitador modificó el segmento \\
\hline
orden & INTEGER & Sí & Orden secuencial en la audiencia \\
\hline
palabras\_json & JSONB & No & Palabras con start, end, confidence \\
\hline
\end{tabular}
\caption{Campos Principales del Modelo Segmento}
\end{table}

\section{Módulo: Canvas y Edición}

Los segmentos se muestran con SpeakerNode (etiqueta del hablante), timestamps y marcas de baja confianza (fondo amarillo si confidence $<$ 0.7). El digitador puede editar texto en línea; al guardar se marca \texttt{editado\_por\_usuario=true}. Las sugerencias son solo reemplazo de palabra (1:1); no se insertan palabras nuevas. Atajos: Tab aceptar sugerencia, Esc rechazar, 1--5 para elegir opción en WordCorrectionPopover.

\section{Módulo: Generación de Acta}

Snapshot del contenido del Canvas + metadatos de audiencia (expediente, juzgado, tipo, fecha, hablantes con roles) $\to$ envío a Claude Sonnet 4 con prompt de formato oficial (Formato A unipersonal / Formato B apelaciones) $\to$ acta en editor separado (ActaEditor) $\to$ edición por digitador $\to$ aprobación por supervisor $\to$ exportación DOCX/PDF.

\section{Módulo: Hablantes y Marcadores}

\textbf{Hablantes:} Asignación de rol (Juez, Fiscal, Defensa, Imputado, Agraviado, etc.) y etiqueta en Canvas por \texttt{speaker\_id}; color por hablante; prioridad de transcripción (crítica, alta, media). API: GET/PUT \texttt{/api/audiencias/\{id\}/hablantes/\{speaker\_id\}}. \\
\textbf{Marcadores:} Creación con nota y \texttt{timestamp\_audio}; BookmarkNode en TipTap; PanelMarcadores con lista cronológica; clic navega al punto en Canvas y en el reproductor de audio. API: POST, GET, DELETE \texttt{/api/audiencias/\{id\}/marcadores}.

% ============================================================================
% CAPÍTULO 6: MODELO DE PRICING Y COSTOS
% ============================================================================
\chapter{Modelo de Pricing y Costos}

\section{Filosofía de Pricing}

El modelo se estructura por niveles mensuales: el cliente paga por el mes en curso, valida el entregable antes de comprometer el siguiente nivel, y no existe compromiso anual obligatorio. La inversión cubre desarrollo, pruebas, despliegue y capacitación básica de cada nivel.

\section{Estructura de Inversión por Nivel}

\begin{table}[H]
\centering
\small
\begin{tabular}{|c|c|c|c|p{3.8cm}|p{2cm}|p{2.2cm}|}
\hline
\rowcolor{primaryblue!20}
\textbf{Nivel} & \textbf{Mes} & \textbf{Sprints} & \textbf{Alcance} & \textbf{Entregables} & \textbf{Inversión} & \textbf{Acumulado} \\
\hline
NIVEL 1 & Mes 1 & 1--5 & MVP & Multi-rol, estructura documento, audios y transcripciones, Canvas, diccionario & \textbf{S/. 1,500} & S/. 1,500 \\
\hline
NIVEL 2 & Mes 2 & 6--10 & Extensión & Mejoras de precisión, batch, acta, exportación DOCX/PDF & S/. 1,500 -- 1,800 & S/. 3,000 -- 3,300 \\
\hline
NIVEL 3 & Mes 3 & 11--15 & Cierre & Usabilidad, corpus opcional, capacitación, soporte & S/. 1,500 -- 1,800 & S/. 4,500 -- 5,100 \\
\hline
\rowcolor{secondarygreen!20}
\multicolumn{5}{|r|}{\textbf{TOTAL SISTEMA COMPLETO:}} & \textbf{15 Sprints} & \textbf{S/. 4,500 -- 5,100} \\
\hline
\end{tabular}
\caption{Inversión por Nivel y Acumulado -- JudiScribe}
\end{table}

\subsection{Qué incluye cada nivel}

Cada inversión mensual incluye:

\begin{itemize}
    \item 5 sprints de desarrollo con funcionalidades implementadas y probadas.
    \item Reuniones de seguimiento (semanal o según acuerdo).
    \item Correcciones y ajustes dentro del alcance del sprint, sin costo adicional.
    \item Documentación técnica actualizada (arquitectura, API, despliegue).
    \item Capacitación básica a usuarios finales del nivel entregado.
    \item Soporte técnico durante el mes de desarrollo.
    \item Despliegue a ambiente de producción (o entrega de artefactos para que el cliente despliegue).
\end{itemize}

Los costos de infraestructura (servidor, PostgreSQL, Redis, Deepgram, Anthropic) son responsabilidad del cliente o se facturan por consumo según acuerdo. Sin costos ocultos por correcciones dentro del alcance definido del sprint; los cambios de alcance se cotizan por separado.

\section{Costos de Infraestructura (Responsabilidad del Cliente)}

\begin{table}[H]
\centering
\small
\begin{tabular}{|p{4.5cm}|p{3.5cm}|p{4cm}|}
\hline
\rowcolor{primaryblue!20}
\textbf{Servicio} & \textbf{Costo Aprox.} & \textbf{Frecuencia} \\
\hline
VPS / servidor (backend, frontend, nginx) & \$60--120 USD & Anual o mensual \\
\hline
PostgreSQL (gestión o DBaaS) & \$0 (local) o \$25+ USD & Mensual si cloud \\
\hline
Redis (broker Celery, caché) & \$0 (mismo servidor) & -- \\
\hline
Deepgram (transcripción) & Por uso (minutos de audio) & Consumo \\
\hline
Anthropic (Claude API) & Por uso (tokens) & Consumo \\
\hline
Dominio y SSL & \$15--25 USD & Anual \\
\hline
\rowcolor{secondarygreen!20}
\textbf{Total estimado (referencial)} & \textbf{\$100--250 USD/año} & + consumo APIs \\
\hline
\end{tabular}
\caption{Infraestructura y Servicios Externos (referencia)}
\end{table}

\section{Opciones de Mantenimiento Post-Desarrollo}

\begin{itemize}
    \item \textbf{Mantenimiento gestionado:} Monitoreo, actualizaciones de seguridad, soporte técnico y corrección de bugs sin costo adicional; no incluye nuevas funcionalidades. Costo mensual a acordar.
    \item \textbf{Cliente autónomo:} Entrega de código fuente, documentación de despliegue y credenciales; el cliente administra su infraestructura.
\end{itemize}

\section{Garantías}

Durante el desarrollo: si el cliente no aprueba los entregables de un sprint, se corrige sin costo adicional hasta la aprobación; los cambios de alcance se cotizan por separado. Post-desarrollo: garantía de corrección de defectos por un período definido tras la aceptación de cada nivel (ej. 1 mes por nivel).

% ============================================================================
% CAPÍTULO 7: REQUERIMIENTOS NO FUNCIONALES
% ============================================================================
\chapter{Requerimientos No Funcionales}

\section{Rendimiento y Tiempo de Respuesta}

\begin{table}[H]
\centering
\small
\begin{tabular}{|p{5cm}|p{3cm}|p{5cm}|}
\hline
\rowcolor{primaryblue!20}
\textbf{Métrica} & \textbf{Objetivo} & \textbf{Justificación} \\
\hline
Latencia de primera transcripción & $<$ 2 s desde inicio de habla & Feedback en tiempo real para el digitador \\
\hline
Diccionario jurídico (check\_segment) & $<$ 50 ms por segmento & No bloquear el flujo de segmentos \\
\hline
WebSocket de transcripción & Timeout 3600 s (proxy) & Audiencias de 2--5 horas sin corte \\
\hline
Generación de acta (Claude) & $<$ 60 s para acta típica & Experiencia fluida al finalizar \\
\hline
Exportación DOCX/PDF & $<$ 10 s & Descarga sin espera excesiva \\
\hline
\end{tabular}
\caption{Métricas de Rendimiento -- JudiScribe}
\end{table}

\section{Disponibilidad y Confiabilidad}

\begin{itemize}
    \item Objetivo de disponibilidad: 99\% mensual en horario de audiencias (ej. 8:00--18:00).
    \item Backups de PostgreSQL con retención configurable (mínimo 7 días). Los archivos WAV se almacenan en disco local o storage externo según configuración.
    \item Recuperación ante fallo: reinicio de servicios con Docker Compose; restauración de BD desde backup si es necesario.
\end{itemize}

\section{Seguridad}

\begin{itemize}
    \item \textbf{Autenticación:} JWT en query para WebSocket (\texttt{?token=}); validación de permiso por audiencia (\texttt{created\_by} o rol admin/supervisor). Access token y refresh token con expiración configurable.
    \item \textbf{Contraseñas:} bcrypt (12 rounds). No se almacenan en claro.
    \item \textbf{Privacidad:} No se envía audio a Claude; solo texto. Los keyterms y el diccionario son datos del sistema, no de terceros.
    \item \textbf{HTTPS} en producción; nginx con TLS termination. Datos sensibles solo en variables de entorno (.env no versionado).
\end{itemize}

\section{Usabilidad}

\begin{itemize}
    \item El digitador debe poder aceptar o rechazar cualquier sugerencia con un solo gesto (Tab aceptar, Esc rechazar).
    \item Indicadores claros: texto provisional (gris) vs confirmado (negro); baja confianza (fondo amarillo, subrayado punteado).
    \item Auto-scroll opcional; se desactiva al hacer scroll manual hacia arriba para editar; reactivación con Ctrl+J (o atajo configurable).
    \item Panel de hablantes siempre visible con indicador de ``hablando ahora''. Reproductor de audio sincronizado con el segmento seleccionado en el Canvas.
\end{itemize}

\section{Escalabilidad}

El sistema está dimensionado para múltiples audiencias concurrentes (una por digitador). Cada sesión de transcripción mantiene una conexión WebSocket a Deepgram y una a el cliente. PostgreSQL y Redis soportan el volumen de segmentos y colas Celery. El procesamiento batch (faster-whisper, Pyannote) puede ejecutarse en un worker con GPU dedicada o en servicio externo.

\section{Compatibilidad}

\begin{itemize}
    \item \textbf{Navegador:} Chrome o Edge (recomendados) para captura de audio (getUserMedia, getDisplayMedia) y WebSocket estable. Firefox y Safari pueden requerir validación adicional.
    \item \textbf{Servidor:} Linux recomendado para despliegue (Docker); Windows Server posible con soporte Docker. Python 3.11, Node 18+ para frontend.
    \item \textbf{Audio:} Formatos de entrada soportados en batch: WAV, MP3, M4A, OGG. Streaming: PCM 16\,kHz mono.
\end{itemize}

\section{Mantenibilidad}

Código modular (backend por api, ws, services, models; frontend por app, components, hooks, lib). Migraciones con Alembic; configuración por variables de entorno. Documentación técnica y README para que otro desarrollador pueda dar soporte o extender el sistema.

% ============================================================================
% CAPÍTULO 7: HISTORIAS DE USUARIO Y CRITERIOS DE ACEPTACIÓN
% ============================================================================
\chapter{Historias de Usuario y Criterios de Aceptación}

\section{Formato de Historias de Usuario}

Las historias siguen el formato: \textbf{Como} [rol], \textbf{Quiero} [acción], \textbf{Para} [beneficio]. Los criterios de aceptación se expresan en escenarios DADO/CUANDO/ENTONCES cuando aplica.

\section{Historias de Usuario -- Digitador (Transcriptor)}

\subsection{HU-DIG-001: Transcribir audiencia en tiempo real}

\textbf{Como} digitador \\
\textbf{Quiero} que el sistema transcriba el audio de la audiencia en tiempo real \\
\textbf{Para} seguir el debate sin tener que escribir todo manualmente

\textbf{Criterios de Aceptación:}
\begin{itemize}
    \item Al iniciar la transcripción, el audio se captura y se envía al servidor; el primer texto aparece en menos de 2 segundos tras el inicio de habla.
    \item Los segmentos se muestran con la etiqueta del hablante (SPEAKER\_00, etc.) y se pueden asignar roles desde el panel de hablantes.
    \item El texto provisional aparece en gris y pasa a negro cuando Deepgram confirma el segmento.
\end{itemize}

\subsection{HU-DIG-002: Corregir palabras con sugerencias del diccionario}

\textbf{Como} digitador \\
\textbf{Quiero} ver sugerencias de corrección para términos jurídicos mal transcritos \\
\textbf{Para} aceptar o rechazar con un solo gesto (Tab/Esc) sin escribir manualmente

\textbf{Criterios de Aceptación:}
\begin{itemize}
    \item Las sugerencias del diccionario (ej. ``sobresemiento'' $\to$ ``sobreseimiento'') se muestran en un popover; Tab acepta la corrección, Esc la rechaza.
    \item Las correcciones son solo reemplazo de palabra (1:1); no se añaden palabras nuevas al texto.
\end{itemize}

\subsection{HU-DIG-003: Editar segmentos sin que la IA los sobrescriba}

\textbf{Como} digitador \\
\textbf{Quiero} que mis ediciones no sean modificadas por la IA \\
\textbf{Para} mantener el control total del texto que ya revisé

\textbf{Criterios de Aceptación:}
\begin{itemize}
    \item Si edito un segmento y guardo, ese segmento queda marcado como \texttt{editado\_por\_usuario}; el sistema no lo reemplaza con propuestas de batch ni con nuevo texto de streaming.
\end{itemize}

\subsection{HU-DIG-004: Generar acta desde el Canvas}

\textbf{Como} digitador \\
\textbf{Quiero} generar el acta oficial con un clic a partir del contenido transcrito \\
\textbf{Para} obtener el borrador formateado según el Poder Judicial sin hacerlo manualmente

\textbf{Criterios de Aceptación:}
\begin{itemize}
    \item Botón ``Generar acta'' envía el contenido del Canvas y metadatos a Claude; el acta generada se carga en el editor de acta para revisión y edición.
\end{itemize}

\subsection{HU-DIG-005: Exportar acta a Word y PDF}

\textbf{Como} digitador (o supervisor) \\
\textbf{Quiero} exportar el acta aprobada a DOCX y PDF \\
\textbf{Para} entregar el documento oficial al juzgado

\textbf{Criterios de Aceptación:}
\begin{itemize}
    \item Exportación DOCX y PDF disponibles cuando el acta está en estado aprobada; el archivo descargado tiene el formato oficial acordado.
\end{itemize}

\section{Historias de Usuario -- Supervisor}

\subsection{HU-SUP-001: Aprobar acta para exportación}

\textbf{Como} supervisor \\
\textbf{Quiero} aprobar formalmente el acta generada por el digitador \\
\textbf{Para} que pueda exportarse como documento oficial

\textbf{Criterios de Aceptación:}
\begin{itemize}
    \item Solo el rol supervisor (o admin) puede ejecutar ``Aprobar acta''; el estado pasa a ``aprobada'' y se registra quién y cuándo aprobó.
\end{itemize}

\section{Priorización MoSCoW}

\begin{table}[H]
\centering
\small
\begin{tabular}{|p{2.2cm}|p{5.5cm}|p{2.2cm}|p{2.2cm}|}
\hline
\rowcolor{primaryblue!20}
\textbf{ID} & \textbf{Historia} & \textbf{Prioridad} & \textbf{Sprint} \\
\hline
HU-DIG-001 & Transcribir en tiempo real & MUST & 1--2 \\
\hline
HU-DIG-002 & Corregir con sugerencias diccionario & MUST & 3--4 \\
\hline
HU-DIG-003 & Editar sin sobrescritura IA & MUST & 4--5 \\
\hline
HU-DIG-004 & Generar acta desde Canvas & MUST & 5, 8 \\
\hline
HU-DIG-005 & Exportar DOCX/PDF & MUST & 9 \\
\hline
HU-SUP-001 & Aprobar acta & MUST & 8 \\
\hline
\end{tabular}
\caption{Priorización de Historias de Usuario}
\end{table}

% ============================================================================
% CAPÍTULO 8: UI/UX Y SISTEMA DE DISEÑO
% ============================================================================
\chapter{UI/UX y Sistema de Diseño}

\section{Principios de Diseño}

\begin{itemize}
    \item \textbf{Control del digitador:} Toda propuesta automática (sugerencia de palabra, texto mejorado) puede aceptarse o rechazarse explícitamente; no hay reemplazo automático sin intervención.
    \item \textbf{Feedback visual inmediato:} Texto provisional vs confirmado, baja confianza resaltada, estado de conexión Deepgram visible en la barra de estado.
    \item \textbf{Atajos de teclado:} Tab/Esc para sugerencias; atajos para frases estándar; Ctrl+J para auto-scroll. Reducen el uso del ratón durante la transcripción.
    \item \textbf{Sincronización audio--texto:} Clic en un segmento del Canvas mueve el reproductor al timestamp correspondiente; el digitador puede verificar lo que oye.
\end{itemize}

\section{Sistema de Diseño}

\subsection{Paleta de Colores}

\begin{table}[H]
\centering
\small
\begin{tabular}{|p{2.8cm}|p{2.2cm}|p{7cm}|}
\hline
\rowcolor{primaryblue!20}
\textbf{Color} & \textbf{Hex} & \textbf{Uso Principal} \\
\hline
Azul primario & \#0066CC & Botones principales, enlaces, títulos \\
\hline
Verde éxito & \#009950 & Confirmaciones, texto correcto, conexión activa \\
\hline
Amarillo alerta & \#FFF3CD & Baja confianza, texto a revisar \\
\hline
Naranja sugerencia & \#E67E22 & Sugerencias del diccionario jurídico \\
\hline
Rojo error & \#CC0000 & Errores, conexión perdida \\
\hline
Gris texto & \#333333 & Texto principal \\
\hline
Fondo claro & \#F5F5F5 & Texto provisional, fondos de panel \\
\hline
\end{tabular}
\caption{Paleta de Colores del Sistema}
\end{table}

\subsection{Tipografía}

Familia de fuentes legible para largas sesiones de lectura (ej. Inter o similar). Tamaño mínimo de 14--16px para el contenido del Canvas. Etiquetas de hablante en negrita y mayúsculas para identificación rápida.

\section{Flujos de Usuario Principales}

\begin{enumerate}
    \item \textbf{Flujo de transcripción en vivo:} Login $\to$ crear/abrir audiencia $\to$ iniciar transcripción $\to$ captura de audio $\to$ segmentos aparecen en el Canvas $\to$ asignar roles a hablantes $\to$ aceptar o rechazar sugerencias $\to$ editar texto si aplica $\to$ detener transcripción.
    \item \textbf{Flujo de corrección:} Clic en palabra de baja confianza o con sugerencia $\to$ se abre WordCorrectionPopover o SuggestionPopover $\to$ Tab aceptar o Esc rechazar (o elegir opción 1--5).
    \item \textbf{Flujo de acta:} Finalizada la audiencia $\to$ botón ``Generar acta'' $\to$ Claude genera borrador $\to$ edición en ActaEditor $\to$ supervisor aprueba $\to$ exportar DOCX o PDF.
\end{enumerate}

\section{Layout de Pantalla}

La vista de transcripción se divide en: (1) \textbf{Canvas} (aprox. 70\% del ancho): editor TipTap con segmentos, hablantes y marcadores; (2) \textbf{Panel derecho} (aprox. 30\%): Panel de hablantes, reproductor de audio, panel de marcadores, panel de diccionario, botón Generar acta. Barra de estado en la parte inferior con conteo de palabras, tiempo transcurrido y estado de conexión.

\section{Indicadores Visuales en el Canvas}

\begin{table}[H]
\centering
\small
\begin{tabular}{|p{3.5cm}|p{3cm}|p{6.5cm}|}
\hline
\rowcolor{primaryblue!20}
\textbf{Elemento} & \textbf{Estilo} & \textbf{Descripción} \\
\hline
Texto provisional & Fondo \#F5F5F5, texto \#999 & is\_final=false de Deepgram \\
\hline
Texto confirmado & Texto negro & is\_final=true \\
\hline
Baja confianza & Fondo \#FFF3CD, subrayado punteado & confidence $<$ 0.7 \\
\hline
Sugerencia jurídica & Subrayado punteado naranja & Corrección del diccionario \\
\hline
Etiqueta hablante & Negrita, mayúsculas, color del rol & SpeakerNode \\
\hline
Marcador & Banderita en margen, tooltip con nota & BookmarkNode \\
\hline
\end{tabular}
\caption{Indicadores Visuales en el Canvas}
\end{table}

\section{Accesibilidad}

Contraste adecuado (WCAG 2.1 nivel AA cuando sea posible), navegación por teclado en formularios y popovers, etiquetas ARIA en controles críticos. El reproductor de audio debe ser operable por teclado.

% ============================================================================
% CAPÍTULO 9: CASOS DE USO DETALLADOS
% ============================================================================
\chapter{Casos de Uso Detallados}

\section{Formato de Casos de Uso}

Cada caso de uso incluye: identificador, nombre, actor principal, precondiciones, postcondiciones, flujo principal y opcionalmente flujos alternativos o de excepción.

\section{CU-001: Transcribir Audiencia en Tiempo Real}

\subsection{Información General}

\begin{table}[H]
\centering
\small
\begin{tabular}{|p{3.5cm}|p{9.5cm}|}
\hline
\rowcolor{primaryblue!20}
\textbf{Atributo} & \textbf{Valor} \\
\hline
ID & CU-001 \\
\hline
Nombre & Transcribir audiencia en tiempo real \\
\hline
Actor Principal & Digitador (transcriptor) \\
\hline
Actores Secundarios & Backend (WebSocket), Deepgram, Claude \\
\hline
Prioridad & Crítica (MUST) \\
\hline
\end{tabular}
\caption{CU-001: Información General}
\end{table}

\subsection{Precondiciones}

\begin{enumerate}
    \item El digitador ha iniciado sesión y tiene permiso sobre la audiencia.
    \item La audiencia está creada con metadatos (expediente, juzgado, tipo, fecha).
    \item Hay fuente de audio disponible (micrófono, salida del sistema o archivo).
\end{enumerate}

\subsection{Postcondiciones (Éxito)}

\begin{enumerate}
    \item El audio se graba en WAV en el servidor; \texttt{audio\_path} y duración actualizados en la audiencia.
    \item Los segmentos transcritos se guardan en PostgreSQL con speaker\_id, texto\_ia, texto\_mejorado, timestamps y orden.
    \item El Canvas muestra el texto en tiempo real con etiquetas de hablante; el digitador puede editar y aceptar o rechazar sugerencias.
\end{enumerate}

\subsection{Flujo Principal}

\begin{enumerate}
    \item El digitador abre la audiencia y pulsa ``Iniciar transcripción''.
    \item El sistema solicita permiso de captura de audio y establece conexión WebSocket con token JWT.
    \item El backend conecta con Deepgram Nova-3 (diarize=true) y comienza a recibir resultados.
    \item El cliente envía chunks de audio (base64 PCM 16\,kHz) cada 250\,ms.
    \item El backend consolida segmentos por speaker, mejora con Claude y aplica el diccionario jurídico; envía segmentos finales y sugerencias al cliente.
    \item El Canvas muestra cada segmento con etiqueta de hablante; el digitador asigna roles desde el panel de hablantes.
    \item Al finalizar, el digitador pulsa ``Detener transcripción''; el WAV se cierra y la audiencia pasa a estado ``transcrita''.
\end{enumerate}

\section{CU-002: Corregir Segmento con Sugerencia del Diccionario}

\subsection{Información General}

\begin{table}[H]
\centering
\small
\begin{tabular}{|p{3.5cm}|p{9.5cm}|}
\hline
\rowcolor{primaryblue!20}
\textbf{Atributo} & \textbf{Valor} \\
\hline
ID & CU-002 \\
\hline
Nombre & Corregir segmento con sugerencia del diccionario \\
\hline
Actor Principal & Digitador \\
\hline
Prioridad & Alta (MUST) \\
\hline
\end{tabular}
\caption{CU-002: Información General}
\end{table}

\subsection{Precondiciones}

El Canvas muestra al menos un segmento con sugerencia (palabra subrayada en naranja). El popover de sugerencia puede mostrarse al hacer hover o al recibir el mensaje \texttt{type: suggestion} por WebSocket.

\subsection{Flujo Principal}

\begin{enumerate}
    \item El sistema detecta una variante errónea en el texto (ej. ``sobresemiento'') y envía sugerencia (``sobreseimiento'').
    \item El digitador ve el popover con original $\to$ sugerido y categoría.
    \item El digitador pulsa Tab para aceptar o Esc para rechazar.
    \item Si acepta, el texto del segmento se actualiza (reemplazo 1:1); si rechaza, el popover se cierra y el texto permanece.
\end{enumerate}

\section{CU-003: Generar Acta desde el Canvas}

\subsection{Información General}

\begin{table}[H]
\centering
\small
\begin{tabular}{|p{3.5cm}|p{9.5cm}|}
\hline
\rowcolor{primaryblue!20}
\textbf{Atributo} & \textbf{Valor} \\
\hline
ID & CU-003 \\
\hline
Nombre & Generar acta desde el Canvas \\
\hline
Actor Principal & Digitador \\
\hline
Actores Secundarios & Claude Sonnet 4 \\
\hline
Prioridad & Crítica (MUST) \\
\hline
\end{tabular}
\caption{CU-003: Información General}
\end{table}

\subsection{Precondiciones}

La audiencia está en estado ``transcrita'' o ``en\_revision''. El Canvas tiene contenido (segmentos con texto).

\subsection{Flujo Principal}

\begin{enumerate}
    \item El digitador pulsa ``Generar acta''.
    \item El sistema toma un snapshot del contenido del Canvas y los metadatos de la audiencia y hablantes.
    \item Se envía a Claude Sonnet 4 con el prompt de formato oficial (Formato A o B según instancia).
    \item Claude devuelve el acta estructurada; se guarda en \texttt{acta.contenido\_llm} y se incrementa la versión.
    \item El acta se carga en el ActaEditor para que el digitador revise y edite; estado ``borrador''.
\end{enumerate}

\section{CU-004: Aprobar Acta y Exportar DOCX/PDF}

\subsection{Información General}

\begin{table}[H]
\centering
\small
\begin{tabular}{|p{3.5cm}|p{9.5cm}|}
\hline
\rowcolor{primaryblue!20}
\textbf{Atributo} & \textbf{Valor} \\
\hline
ID & CU-004 \\
\hline
Nombre & Aprobar acta y exportar DOCX/PDF \\
\hline
Actor Principal & Supervisor (aprobación); Digitador o Supervisor (exportación) \\
\hline
Prioridad & Crítica (MUST) \\
\hline
\end{tabular}
\caption{CU-004: Información General}
\end{table}

\subsection{Precondiciones}

Existe un acta en estado ``borrador'' o ``en\_revision''. El usuario que aprueba tiene rol supervisor o admin.

\subsection{Flujo Principal}

\begin{enumerate}
    \item El supervisor revisa el acta en el ActaEditor y pulsa ``Aprobar acta''.
    \item El sistema cambia el estado a ``aprobada'' y registra \texttt{aprobada\_por} y \texttt{aprobada\_at}.
    \item El digitador o supervisor pulsa ``Exportar DOCX'' o ``Exportar PDF''.
    \item El backend genera el archivo (python-docx / weasyprint) y devuelve la descarga; se registra la acción en audit\_log.
\end{enumerate}

% ============================================================================
% CAPÍTULO 10: PLAN DE PRUEBAS
% ============================================================================
\chapter{Plan de Pruebas}

\section{Estrategia de Pruebas}

\begin{itemize}
    \item \textbf{Unitarias:} Servicios de diccionario (tokenización, fuzzy match), consolidación de segmentos (reglas de completitud), text\_processing (detect\_question, clean\_transcript).
    \item \textbf{Integración:} API REST (audiencias, segmentos, hablantes, actas, exportación); WebSocket de transcripción con mock de Deepgram o pruebas contra entorno de staging.
    \item \textbf{E2E:} Flujo completo: login $\to$ crear audiencia $\to$ iniciar transcripción (audio simulado o corto) $\to$ recibir segmentos $\to$ editar $\to$ generar acta $\to$ aprobar $\to$ exportar DOCX/PDF. Herramientas: Playwright o Cypress.
\end{itemize}

\section{Pruebas Unitarias Clave}

\begin{itemize}
    \item \texttt{LegalDictionary.check\_segment}: dado un texto con variantes erróneas conocidas, retorna la lista de correcciones en menos de 50\,ms.
    \item \texttt{detect\_question}: textos que terminan en ``?'' o empiezan por palabra interrogativa devuelven True.
    \item Reglas de consolidación: buffer que termina en ``que'' con más de 5 palabras se considera incompleto; respuestas ``sí'', ``no'', ``niego'' se consideran completas.
\end{itemize}

\section{Pruebas End-to-End (E2E)}

\begin{itemize}
    \item \textbf{E2E-1:} Login con rol transcriptor $\to$ crear audiencia $\to$ abrir Canvas $\to$ iniciar transcripción con archivo de audio corto pregrabado $\to$ verificar que aparecen segmentos y que el WAV se guarda.
    \item \textbf{E2E-2:} Dado Canvas con segmentos $\to$ generar acta $\to$ verificar que el acta aparece en el editor $\to$ aprobar (como supervisor) $\to$ exportar DOCX $\to$ verificar descarga.
\end{itemize}

\section{Pruebas de Rendimiento}

\begin{itemize}
    \item Latencia del diccionario: medir tiempo de \texttt{check\_segment} para segmentos de 50--200 palabras; objetivo $<$ 50 ms.
    \item WebSocket: mantener conexión activa durante al menos 1 hora con envío periódico de chunks; verificar que no hay desconexión por timeout del proxy.
\end{itemize}

\section{Pruebas de Usabilidad con Usuarios Reales}

Sesiones con 2--3 digitadores del Distrito Judicial: tarea de transcribir un audio de prueba de 5--10 minutos, corregir al menos 3 sugerencias y generar el acta. Medir tiempo y número de errores; recoger feedback sobre claridad de indicadores y atajos.

\section{Criterios de Aceptación por Sprint}

Para aprobar un sprint: (1) Todas las funcionalidades del sprint implementadas y probadas; (2) Tests E2E de flujos críticos del sprint en verde; (3) Demo con el responsable del proyecto sin errores bloqueantes; (4) Documentación de cambios relevante actualizada.

% ============================================================================
% CAPÍTULO 11: PLAN DE CAPACITACIÓN
% ============================================================================
\chapter{Plan de Capacitación}

\section{Estrategia de Capacitación}

La capacitación se realiza por rol y por nivel. Al finalizar Nivel 1 (MVP), se capacita a los digitadores en el uso del Canvas, sugerencias y generación de acta. Al finalizar Nivel 2, se incluye revisión de propuestas batch, aprobación de acta y exportación. Se utiliza el sistema con datos reales o de prueba representativos, no solo datos ficticios.

\section{Plan de Capacitación por Rol}

\subsection{Capacitación para el Digitador (Transcriptor)}

\textbf{Duración total:} 2--3 horas presenciales + acompañamiento en la primera semana.

\textbf{Módulos:}
\begin{enumerate}
    \item \textbf{Navegación y creación de audiencia (20 min):} Login, creación de audiencia con todos los campos del encabezado (expediente, juzgado, tipo, instancia, fecha, sala, participantes). Estructura clara del documento.
    \item \textbf{Transcripción en vivo (45 min):} Iniciar/detener transcripción, selección de fuente de audio, interpretación del Canvas (texto provisional vs confirmado, baja confianza). Panel de hablantes: asignación de roles (Juez, Fiscal, Defensa, Imputado, etc.) y etiquetas.
    \item \textbf{Corrección y sugerencias (30 min):} Aceptar o rechazar sugerencias del diccionario (Tab/Esc). WordCorrectionPopover para palabras de baja confianza. Edición manual de segmentos; regla de no-sobreescritura.
    \item \textbf{Marcadores y frases estándar (15 min):} Creación de marcadores, atajos de frases estándar.
    \item \textbf{Generación de acta y exportación (30 min):} Generar acta desde el Canvas, revisar en el editor de acta, flujo de aprobación (supervisor) y exportación DOCX/PDF.
\end{enumerate}

\subsection{Capacitación para el Supervisor}

\textbf{Duración:} 1 hora.

Revisión de actas generadas por los digitadores, edición menor si aplica, aprobación formal y exportación. Diferencias de permisos respecto al digitador.

\subsection{Capacitación para el Administrador}

\textbf{Duración:} 1--1.5 horas.

Gestión de usuarios (roles, activación), configuración del diccionario jurídico y keyterms si la aplicación lo expone, despliegue básico y variables de entorno.

\section{Evaluación Post-Capacitación}

\begin{itemize}
    \item \textbf{Digitador:} Realizar una transcripción de prueba (audio de 5--10 min), asignar roles a al menos 2 hablantes, aceptar o rechazar al menos 3 sugerencias y generar el acta sin asistencia. Tiempo objetivo: coherente con el flujo real.
    \item \textbf{Supervisor:} Aprobar un acta de prueba y exportar a DOCX y PDF correctamente.
\end{itemize}

\section{Soporte Post-Capacitación}

Durante la primera semana tras el cierre del proyecto (Sprint 15), soporte prioritario para incidencias de uso y configuración. Canal de contacto (correo o canal interno) y tiempo de respuesta objetivo documentado (ej. respuesta en menos de 4 horas para incidencias críticas).

% ============================================================================
% CAPÍTULO 12: SOPORTE Y MANTENIMIENTO
% ============================================================================
\chapter{Soporte y Mantenimiento}

\section{Período de Garantía}

\begin{table}[H]
\centering
\small
\begin{tabular}{|p{2.5cm}|p{2.5cm}|p{8cm}|}
\hline
\rowcolor{primaryblue!20}
\textbf{Nivel} & \textbf{Duración} & \textbf{Vigencia} \\
\hline
Nivel 1 & 1 mes & Desde la aprobación del Sprint 5 \\
\hline
Nivel 2 & 1 mes & Desde la aprobación del Sprint 10 \\
\hline
Nivel 3 & 1 mes & Desde la aprobación del Sprint 15 \\
\hline
\end{tabular}
\caption{Período de Garantía por Nivel}
\end{table}

Corrección de defectos sin costo adicional durante el período de garantía. Los cambios de alcance se cotizan por separado.

\section{Niveles de Soporte y SLA}

\begin{table}[H]
\centering
\small
\begin{tabular}{|p{2.2cm}|p{4.5cm}|p{2.2cm}|p{3.5cm}|}
\hline
\rowcolor{primaryblue!20}
\textbf{Severidad} & \textbf{Descripción} & \textbf{Respuesta} & \textbf{Resolución} \\
\hline
Crítico & Sistema caído o transcripción no funciona & 1 hora & 4 horas \\
\hline
Alto & WebSocket se desconecta, acta no se genera & 4 horas & 1 día \\
\hline
Medio & Función secundaria no disponible, hay alternativa & 1 día & 3 días \\
\hline
Bajo & Problema cosmético, consulta general & 2 días & 1 semana \\
\hline
\end{tabular}
\caption{SLA de Soporte Post-Garantía}
\end{table}

\section{Canales de Soporte}

\begin{enumerate}
    \item \textbf{Email / canal interno:} Para incidencias no críticas y consultas. Respuesta según SLA.
    \item \textbf{Contacto prioritario:} Para severidad crítica durante el período de garantía (teléfono o canal acordado).
\end{enumerate}

\section{Mantenimiento Preventivo}

\textbf{Actividades recomendadas (mensual o trimestral):} Actualización de dependencias con vulnerabilidades de seguridad; revisión de logs y tiempos de respuesta del API y del WebSocket; verificación de backups de PostgreSQL y de archivos WAV; revisión de permisos y cuentas de usuario activas.

\section{Documentación Entregada al Finalizar el Proyecto}

\begin{enumerate}
    \item \textbf{Documentación técnica:} Arquitectura, modelo de datos, documentación de API (OpenAPI/Swagger), guía de despliegue (Docker Compose, nginx) y variables de entorno.
    \item \textbf{Manuales de usuario:} Por rol (digitador, supervisor, administrador) con capturas de pantalla de los flujos principales.
    \item \textbf{Código fuente:} Repositorio con historial y README. Credenciales y claves (Deepgram, Anthropic) gestionadas de forma segura (no en el repositorio).
\end{enumerate}

% ============================================================================
% CAPÍTULO 12: CONCLUSIONES Y PRÓXIMOS PASOS
% ============================================================================
\chapter{Conclusiones y Próximos Pasos}

\section{Hallazgos Principales}

JudiScribe aborda el cuello de botella de la transcripción manual en audiencias judiciales mediante IA (Deepgram + Claude) manteniendo el control total del digitador. La consolidación por speaker y el diccionario jurídico reducen errores en términos legales sin añadir palabras no pronunciadas. La regla de no-sobreescritura garantiza que el trabajo del digitador nunca se pierde.

\section{Recomendaciones}

\begin{itemize}
    \item Mantener el diccionario jurídico (\texttt{legal\_terms.json}) actualizado con los términos más problemáticos reportados por los digitadores.
    \item Evaluar la incorporación del pipeline de corpus (Sprint 12) para que el sistema aprenda de las correcciones y mejore keyterms y variantes con el tiempo.
    \item Configurar monitoreo de latencia y disponibilidad del WebSocket y de los servicios externos (Deepgram, Anthropic).
\end{itemize}

\section{Próximos Pasos}

\begin{itemize}
    \item Kickoff: definición de keyterms y diccionario inicial con el Distrito Judicial de Cusco.
    \item Sprint 1--5: desarrollo del MVP según este documento.
    \item Demos semanales para validación con usuarios finales.
    \item Nivel 2: activación del pipeline batch cuando se disponga de GPU o servicio externo.
    \item Nivel 3: cierre, capacitación y soporte post-lanzamiento.
\end{itemize}

% ============================================================================
% ANEXOS
% ============================================================================
\chapter{Anexos}

\section{Anexo A: Glosario de Términos}

\begin{itemize}
    \item \textbf{Segmento:} Unidad de texto transcrito con un speaker, timestamps y opcionalmente palabras con confianza. Corresponde a una o varias frases consolidadas del mismo hablante.
    \item \textbf{Consolidación:} Agrupar varios resultados finales de Deepgram del mismo speaker en una sola frase antes de mejorar y guardar, para evitar cortes en mitad de oración.
    \item \textbf{Editado por usuario:} Flag que impide que la IA modifique ese segmento; ningún proceso automático (streaming, batch, LLM) puede sobrescribirlo.
    \item \textbf{Keyterms:} Términos enviados a Deepgram en la URL de conexión para mejorar el reconocimiento de vocabulario jurídico (límite 100 por sesión).
    \item \textbf{Texto mejorado:} Salida de Claude sobre el texto crudo de Deepgram: puntuación, mayúsculas (Juez, Fiscal, Señoría), signos de interrogación, sin inventar contenido.
    \item \textbf{Provisional:} Segmento aún no confirmado por Deepgram (is\_final=false); puede cambiar hasta que Deepgram envíe el resultado final.
    \item \textbf{Sprint:} Unidad de desarrollo con funcionalidades concretas entregables y verificables; típicamente 1--2 semanas.
\end{itemize}

\section{Anexo B: Referencias}

\begin{itemize}
    \item Readme.md (Guía Maestra de Implementación JudiScribe) en la raíz del proyecto.
    \item CLAUDE.md: reglas del proyecto y contexto para IA.
    \item Plan de revisión ``Revisión JudiScribe y modelo de aprendizaje'': mejoras de precisión (no añadir palabras), autocompletado solo como corrección 1:1, corpus y tokenización para que el sistema aprenda. Las mejoras de ese plan se distribuyen en Nivel 2 (Sprint 6: reglas de precisión) y Nivel 3 (Sprint 12: corpus e indexación).
    \item Especificación de API REST y WebSocket en la documentación del backend (FastAPI puede exponer OpenAPI/Swagger).
    \item Documentación de Deepgram (streaming, diarization, keyterms) y de Anthropic (Claude API).
\end{itemize}

\section{Anexo C: Referencias Normativas}

Normativa del Poder Judicial del Perú aplicable a actas de audiencia y formato oficial del Distrito Judicial de Cusco. Código Procesal Penal y lineamientos de documentación de audiencias (consultar con el cliente para referencias exactas).

\section{Anexo D: Control de Versiones del Documento}

\begin{table}[H]
\centering
\small
\begin{tabular}{|c|p{3cm}|p{5cm}|}
\hline
\rowcolor{primaryblue!20}
\textbf{Versión} & \textbf{Fecha} & \textbf{Descripción} \\
\hline
1.0 & Febrero 2026 & Versión inicial; estructura por niveles y sprints; Nivel 1 MVP (multi-rol, estructura documento, guardado audios/transcripciones); Nivel 2 con mejoras de precisión y batch; costos (Nivel 1 S/. 1,500); casos de uso, UI/UX, pruebas, capacitación, soporte, firmas. \\
\hline
\end{tabular}
\caption{Control de Versiones del Informe}
\end{table}

% ============================================================================
% CAPÍTULO 15: FIRMAS DE APROBACIÓN
% ============================================================================
\chapter{Firmas de Aprobación}

\section{Declaración de Conformidad}

El presente documento de captura de requerimientos describe el sistema JudiScribe para transcripción judicial en tiempo real del Distrito Judicial de Cusco, Perú. Las especificaciones, costos y plazos descritos están sujetos a la aprobación formal de las partes mediante la firma de este documento y del acuerdo o contrato de servicios correspondiente a cada nivel.

\section{Firmas}

\vspace{2cm}

\begin{table}[H]
\centering
\begin{tabular}{p{7cm}p{7cm}}
\textbf{CLIENTE} & \textbf{PROVEEDOR} \\[0.5cm]
& \\[0.5cm]
\rule{6cm}{0.4pt} & \rule{6cm}{0.4pt} \\
Nombre: & \\
Cargo: & Cargo: \\
Institución: Poder Judicial -- Distrito Judicial de Cusco & Empresa: \\
Fecha: \_\_\_/\_\_\_/\_\_\_\_ & Fecha: \_\_\_/\_\_\_/\_\_\_\_ \\
\end{tabular}
\end{table}

\vspace{2cm}

\section{Control de Aprobaciones por Nivel}

\begin{table}[H]
\centering
\begin{tabular}{|p{2.5cm}|p{4cm}|p{4cm}|p{3cm}|}
\hline
\rowcolor{primaryblue!20}
\textbf{Nivel} & \textbf{Fecha de Aprobación} & \textbf{Firma del Cliente} & \textbf{Estado} \\
\hline
Nivel 1 (MVP) & & & Pendiente \\
\hline
Nivel 2 & & & Pendiente \\
\hline
Nivel 3 & & & Pendiente \\
\hline
\end{tabular}
\caption{Control de Aprobaciones por Nivel}
\end{table}

\vspace{2cm}

\begin{center}
\textbf{\Large FIN DEL DOCUMENTO}

\vspace{1cm}

\textit{Documento de Captura de Requerimientos}\\
\textit{JudiScribe -- Sistema de Transcripción Judicial en Tiempo Real}\\
\textit{Distrito Judicial de Cusco, Perú}

\vspace{1cm}

\textit{Versión 1.0 -- Febrero 2026}
\end{center}

\end{document}
